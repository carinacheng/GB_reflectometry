\documentclass[12pt,preprint]{aastex}
\usepackage[margin=1in]{geometry}
\usepackage{float,amsmath}
%\usepackage{titlesec} %used to format titles
\usepackage{graphicx} %for handling figures
%\usepackage[none]{hyphenat} %disallows hyphenated words


\begin{document}

\title{HERA Dish Reflectometry} 
\author{Zaki Ali, Carina Cheng, Aaron Parsons, Nipanjana Patra}
\maketitle

\section{Introduction}

There are several different sources of instrumental chromaticity for radio
telescope systems that can result in non-ideal performances and unwanted
systematic effects in measured data. One such non-ideality is the mismatch
between the impedance of free space and the antenna and transmission line,
which results in a partial coupling of the sky signal into the antenna while
the rest is reflected into space. For any reflector antenna, such as the HERA
dish antenna, this signal illuminates the reflector and part of it reflects
back and forth several times in between the dish feed and the vertex of the
dish.  Such reflections generate multiple reduced strength copies of the
incident sky signal at various delays, and this produces spurious correlations
in the visibilities of interferometric data. 

The design specification of HERA elements is such that the amplitude of the
signal that arrives at the feed at a delay of $60ns$ (after multiple
reflections off the dish) should be reduced by $60dB$ relative to the first
incident signal at the feed. We aim to verify this by carrying out reflectometry measurements at the HERA antenna prototype in Green Bank, WV.
Understanding the nature of antenna reflections in the HERA dish is of the
utmost importance in characterizing the performance of the dish. As HERA
progresses as an experiment, it is necessary to build optimal dishes that aim
to minimize the challenges of chromaticity in our quest for the Epoch of
Reionization.

\begin{figure}
\centering
\includegraphics[totalheight=0.3\textheight]{plots/reflection_cartoon.png}
\caption{The blue solid lines represent an original sky signal entering the
feed. A small percentage of it (dashed blue) is reflected off the dish, and it
is these reflections that we are concerned about. In our measurements however,
the reflections measured contain most of the original pulse signal (solid red),
so it is crucial to adjust for this difference in our analysis.}
\label{fig:cartoon}
\end{figure}

\begin{figure}
\centering
\includegraphics[totalheight=0.4\textheight]{plots/frequency_amp_phase_fullbw.png}
\caption{Amplitude and phase of the measured return loss. Colored dashed lines
mark three different frequency bands: $140-160MHz$, $100-200MHz$, and
$50-250MHz$.}
\label{fig:freq}
\end{figure}

\section{Theory}{\label{sec:theory}}

In practice, a HERA dish receives signal from the sky\footnote{All astronomical
telescopes operate in this way.}. Plane waves incident on the parabolic dish
are focussed at the feed with focal height $l$. For a well-designed feed, one
that matches the impedance of free space\footnote{The impedance of free space
is $Z_{0} = \mu_{0}c_{0} = \frac{1}{\epsilon_{0}c_{0}} \approx 377\Omega $.},
most of the signal will enter the system while only
a small percentage will be reflected back towards the dish for a secondary
reflection into the feed (blue arrows in Figure \ref{fig:cartoon}). In the
following discussion, we consider the reflection off the feed and the
subsequent reflection off the dish as one reflection.

Quantitatively, if the incident power from the sky signal is $P_{sky}$, the feed
reflection coefficient is $\Gamma_{a}$, and the dish reflection
coefficient is $\Gamma_{d}$, then the net power entering the feed after an
$n^{th}$ reflection off the feed and the dish is:

\begin{equation}\label{eqn:series1}
P_{in} =  P_{sky}(1-\Gamma_{a})\Gamma_{d} [1+ \Gamma_{a}\Gamma_{d} e^{i\phi}+ (\Gamma_{a}\Gamma_{d})^2e^{i2\phi}+ ....+ (\Gamma_{d}\Gamma_{n})^{n}e^{in\phi}]
\end{equation}

where, $\phi = 2l({2\pi \over c})f$ is the propagation delay of a lightwave of frequency $f$ due to a reflection over a focal distance $l$. 
Therefore:
\begin{eqnarray}\label{eqn:ratio1}
{P_{in} \over P_{sky} } & = & (1-\Gamma_{a})\Gamma_{d} [1+ \Gamma_{a}\Gamma_{d} e^{i\phi}+ (\Gamma_{a}\Gamma_{d})^2e^{i2\phi}+ ....+ (\Gamma_{d}\Gamma_{n})^{n}e^{in\phi}] \nonumber\\
      & = & (1-\Gamma_{a})\Gamma_{d} {1-(\Gamma_{d}\Gamma_{a}e^{i\phi})^{n} \over 1-\Gamma_{d} \Gamma_{a} e^{i\phi} } 
\end{eqnarray}

The ratio in Equation \ref{eqn:ratio1} quantifies the amount of power received
with respect to the incident sky power. Realistically, the incident sky power
is not easily quantifiable, but it is a quantity we need to know to accurately
characterize reflections.  


%Reflections cause the sky signal to appear at various delays. We aim to measure
%the relative signal strength at those various delays, referred to as the
%``delay spectrum" hereafter, to estimate levels of reflections in between the
%feed and the dish apex.
 
%We employ the reciprocity theorem for antennas in our experimental set-up.
%More specifically, our experimental set-up differs from measuring sky signal
%reflections since we send a pulse that is to be transmitted from the feed
%before receiving it again. This results in a reversed situation - most of our
%original signal is transmitted from the feed to be reflected from the dish,
%while a small percentage reflects back down the cable (red arrows in Figure
%\ref{fig:cartoon}). Therefore, since our measurement is carried out in
%transmitting mode while an observation would be carried out in receiving mode,
%we correct our results in order to map them to represent real observations.

Therefore, in our experimental set-up, instead of using sky signal, we employ
our feed as a transmitter and transmit a pulse. If the initial pulse is a broadband signal,
$P_{tr}$, sent to the feed antenna via a $75m$ long cable by a vector network
analyser (VNA), a delay domain measurement of the system is accomplished by
measuring the complex return loss of the feed. When the signal is incident on
the feed, part of the incident power ($\Gamma_{a}$ ) is reflected back to the
measuring device (dashed red arrows in Figure \ref{fig:cartoon}) and
$(1-\Gamma_{a})$ is radiated by the feed (solid red arrows in Figure
\ref{fig:cartoon}). The signal radiated by the feed illuminates the dish, and
the signal incident at the dish vertex is reflected by the dish and returns to
the feed. [XXX All the signal incident on the dish is reflected, but we are
only considering radiation from the vertex]. This incident signal is now
reflected back and forth in between the feed and the dish much like the sky
signal reflection discussed previously.  Hence, if $P_{r}$ is the power
incident back on the feed for the first time then the reflected power $P_{ref}$
back into the VNA would be:

\begin{equation}\label{eqn:series2}
P_{ref} =  P_{r}(1-\Gamma_{a}) \Gamma_{d}[1+ \Gamma_{a}\Gamma_{d} e^{i\phi}+ (\Gamma_{a}\Gamma_{d})^2e^{i2\phi}+ ....+ (\Gamma_{d}\Gamma_{n})^{n}e^{in\phi}]
\end{equation}
 
Once again, note that we consider one reflection from the feed and its subsequent reflection from the dish as one reflection in total. Equation \ref{eqn:series2} is similar to Equation \ref{eqn:series1}, with different incident powers.

Recall that $P_{r}$ is the initial power that is incident back on the feed, which is just the feed radiated power reflected off the dish:
 
\begin{equation}
P_{r}= \Gamma_{d}(1-\Gamma_a) P_{tr}
\end{equation}

Also note that the first reflection of the signal sent by the VNA occurs at the antenna end. Hence the total returned power $P_{ret}$, to the VNA  would be:

\begin{eqnarray}
P_{ret} & = & \Gamma_{a}P_{tr} \nonumber\\ 
 & + &   \Gamma_{d}^{2}(1-\Gamma_a) P_{tr}(1-\Gamma_{a}) [1+ \Gamma_{a}\Gamma_{d} e^{i\phi}+  ....+ (\Gamma_{d}\Gamma_{n})^{n}e^{in\phi}]\nonumber\\
 \end{eqnarray}
 
Simplifying:
 
  \begin{eqnarray}\label{eqn:ratio2}
 {P_{ret} \over P_{tr} } & = & \Gamma_{a}
  +  \Gamma_{d}^{2}(1-\Gamma_a)^{2}  {1-(\Gamma_{d}\Gamma_{a}e^{i\phi})^{n} \over 1-\Gamma_{d} \Gamma_{a}e^{i\phi} } \nonumber\\
\end{eqnarray}

The ratio in Equation \ref{eqn:ratio2} is the returned power to the VNA with
respect to the transmitted power sent by the VNA. It is identical to the sky
observation case in Equation \ref{eqn:ratio1} but differs by two factors. The
first factor corresponds to an additive amplitude difference arising from
$\Gamma_{a}$, which physically accounts for the initial reflection at the feed.
The second difference is a multiplicative term which informs us about the first
reflection. Both of these terms need to be corrected for in order to relate our
measurements to real observations.
%Therefore, because our measurement is carried out in transmitting mode
%while a sky observation would be carried out in receiving mode, we correct our
%results in order to map them real observations.

The VNA measures the magnitude and phase of the quantity ${P_{ret}\over P_{tr}}$
as a function of frequency as shown in Figure \ref{fig:freq}. In our measurement
set-up, the first reflection occurs at the antenna terminal $\Gamma_{a}$, so
$({P_{ret} \over P_{tr} }  - \Gamma_{a}) $ gives an estimate of the delay
spectrum of the sky signal. In delay domain, the relative signal strength at
zero delay represents the factor $\Gamma_{a}$ while the signal strength at any
other delay represents any delayed signal that enters the feed after being
reflected from the feed surroundings. 

[XXXAnother bit about corrections at low and high delays needs to be in here. 


\section{Methodology}{\label{sec:methods}}

Our reflectometry measurements are made using a prototype HERA dish (Figure
\ref{fig:heradish}) at NRAO in Green Bank, WV. The dish is a $14\,m$ diameter
parabolic reflector structurally supported with 3 telephone poles. The
reflective material is made up of wire mesh that is attached to PVC
pipes, forming the parabolic shape of the dish. With the current iteration of
the HERA dish, the feed consists of a PAPER dipole encased in a cylindrical cage
encompassing the backplane. The PAPER feed and the backplane (which is aimed at
preventing feed-to-feed interaction between neighboring dishes) is raised and
lowered by a three-pulley system. The focal height of the dish is $4.5m$
($\sim{14.76}$ft).  


%The heights measured in our experiment were made from
%the balun to the top of the concrete hub. However, the focal height should be
%measured from the backplane to point where the wire mesh would intersect the
%concrete hub at the vertex of the dish. To account for this height discrepency,
%we add 1.96 ft to all of our measured heights. These are the heights quoted in
%the plots.[XXX not done yet but will be done]
Our measurements are made with a FieldFox in VNA mode. In this mode, a pulse
is generated in the FieldFox and sent through a $75ft$ $50\Omega$ cable that
connects to the feed with a 4:1 passive balun. The return loss as a function of
frequency, from 50 to $500MHz$, is saved.  Both the amplitude of the relative
power and phase information are saved.  The frequency and delay resolution of
the measurements is $\Delta\nu = 0.44 Mhz$ and $\Delta{t}=2.22ns$, respectively.
In addition, we note that the round trip of a reflection from feed to the dish
is $9m$, which corresponds to a delay of $30ns$.



%Feed heights quoted in our measurements represent the distance from the balun to
%the top of the central concrete hub. However, the actual focal height
%of the dish represents the distance from the backplane of the feed to the dish's
%wire mesh, which intersects the concrete hub between the ground and the top of
%the hub. A 
%XXX get discrepancy distance from DaveD. (1.7 ft from the bottom plate of
%sandwich to the top of the the cage and .26 ft from the top of the concrete hub
%to the middle where the parabola starts)

%need to add in third : Antenna feed focal length 4.5 m which corresponds to a
%signal propagation delay of 0.3 ns. Hence, subsequent reflections from the dish
%vertex are expected to be at a delay which is integral multiple of 0.3 nS.

In order to make our delay domain return loss measurements, we Fourier transform
the frequency domain data we inherently measure with the VNA. However, since  we
are Fourier transforming a finite data series, and therefore a series multiplied
by a square window function, we are actually convolving the Fourier transform of
our measurements with a $sinc$ function. This results in excess power at high
delays due to the sidelobes of the $sinc$ function. To minimize the sidelobes we
 must use an appropriate window function before taking the Fourier transform.
We have chosen a Hamming window for our analysis. The effectiveness of this
window function compared to others is illustrated in Figure
\ref{fig:window}. 

\begin{figure}
\centering
\includegraphics[trim={2cm 20cm 30cm 15cm},clip, totalheight=0.3\textheight]{plots/heradish.jpg}
\caption{HERA dish and feed at the Green Bank NRAO site.}
\label{fig:heradish}
\end{figure}



\begin{figure}
\centering
\includegraphics[totalheight=0.4\textheight]{plots/bh_vs_sq.png}
\caption{Delay plot produced for the PAPER bandwidth ($100MHz-200MHz$) using four different window functions: Blackman-Harris, Hamming, Hanning, and square.}
\label{fig:window}
\end{figure}

As mentioned in Section \ref{sec:theory}, there is a mis-match in amplitude
between the reflections that we measure (originating from the FieldFox pulse)
and reflections produced by sky signal. The reflections that we measure (at high
delays) must be lowered by a factor to represent weaker reflections that would
occur after most of the sky signal is received by the feed. For our
compensation, we multiply our entire delay spectrum by its DC component [XXX
what factor is this in the above equations. Why can we ignore the other factor].
We note that this correction is only accurate at high delays where our
reflections of interest occur. At low delays, our spectrum amplitude should be
increased to represent the original sky signal, but we do not apply this
correction because it is not relevant to our analysis.

\section{Results}

Figure \ref{fig:freq} shows the return loss for a frequency bandwidth of $50$ to
$500MHz$. This measurement was taken with the feed suspended at $13.96ft$, which
was our closest measurement to the actual focal height.  Because the return
loss is the ratio of the power received to the power transmitted, higher
reflections can clearly be seen outside of the PAPER bandwidth. This is not
surprising, since the feed is tuned specifically for PAPER. The return loss
minima are locations where our feed is well-matched to free space.

In Figure \ref{fig:3bands}, the return loss we measure is plotted versus delay for three
chosen bandwidths: the HERA bandwidth, the PAPER bandwidth, and a typical power
spectra bandwidth when using a Hamming window function. A fourth measurement is also plotted using the PAPER bandwidth. This measurement was taken a year ago using a prototype HERA dish in Berkeley, and one of the main differences was that the feed was not encompassed in a cage. From the plot, it is again
shown that the reflections are minimized for the PAPER bandwidth. It appears that the cage surrounding the feed increases the level of reflections.

\begin{figure}
\centering
\includegraphics[totalheight=0.4\textheight]{plots/delay3_window.png}
\caption{XXX temporary caption explaining above: All delay plots use a Hamming window function. 3 of them are our measurements for 3 different bandwidths. Two are Daisy's old measurements (feed in dish without cage) - one (ORANGE) is the FFT that the VNA spits out (using a bandwidth of $50MHz-1000MHz$) and one is her raw frequency data (GREY) analyzed by our pipeline (FFT done in our script, using a bandwidth of $100MHz-200MHz$).}
%\caption{Delay plots produced using a Hamming window function for 4 different frequency bandwidths: $50MHz-250MHz$ (HERA bandwidth), $100MHz-200MHz$ (PAPER bandwidth), $140MHz-160MHz$ (typical power spectrum bandwidth), and $100MHz-200MHz$ (data taken at an earlier time of the feed without the cage). The black dashed lines illustrate our ``60 by 60" specification.}
\label{fig:3bands}
\end{figure}

Figure \ref{fig:elevator} is again a delay plot of the return loss, but for
four different feed suspension heights. We use the PAPER bandwidth and note
that the measurements are near identical at low delays, implying that low delay
reflections are caused primarily by reflections within the feed cage. However,
at higher delays we notice discrepancies between the different heights.

\begin{figure}
\centering
\includegraphics[totalheight=0.4\textheight]{plots/delay_heights_paper.png}
\caption{Delay plots produced using a Hamming window function for 4 different feed heights and the PAPER bandwidth ($100MHz-200MHz$). The black dashed lines illustrate our ``60 by 60" specification.}
\label{fig:elevator}
\end{figure}

Finally, Figure \ref{fig:outofthedish} presents measurements taken of the feed
away from the dish. Echosorb is placed under the feed for some of the
measurements, with the expectation that it will prevent any reflections off the
ground. Measurements are also taken of the feed inside its metal cage in
various configurations. It is shown that the feed performs best without the cage and with the absorber. 

\begin{figure}
\centering
\includegraphics[totalheight=0.4\textheight]{plots/delay_feed.png}
\caption{Delay plots produced using a Hamming window function for different lone feed configurations and the PAPER bandwidth ($100MHz-200MHz$). The black dashed lines illustrate our ``60 by 60" specification.}
\label{fig:outofthedish}
\end{figure}


\section{Conclusion}


\end{document}
